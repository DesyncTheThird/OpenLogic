% Part: turing-machines
% Chapter: undecidability
% Section: universal-tm

\documentclass[../../../include/open-logic-section]{subfiles}

\begin{document}

\olfileid{tur}{und}{uni}
\olsection{Universal Turing Machines}

In \olref[enu]{sec} we discussed how every Turing machine can be
described by a finite sequence of integers. This sequence encodes the
states, alphabet, start state, and instructions of the Turing machine.
We also pointed out that the set of all of these descriptions is
!!{enumerable}. Since the set of such descriptions is !!{denumerable},
this means that there is !!a{surjective} function from~$\Nat$ to
these descriptions. Such !!a{surjective} function can be obtained, for
instance, using Cantor's zig-zag method.  It gives us a way of
enumerating all (descriptions) of Turing machines. If we fix one such
enumeration, it now makes sense to talk of the $1$st, $2$nd, \dots,
$e$th Turing machine.  These numbers are called \emph{indices}.

\begin{defn}
If $M$~is the $e$th Turing machine (in our fixed enumeration), we
say that $e$~is an \emph{index} of~$M$. We write $M_e$ for the $e$th
Turing machine.
\end{defn}

A machine may have more than one index, e.g., two descriptions of~$M$
may differ in the order in which we list its instructions, and these
different descriptions will have different indices.

Importantly, it is possible to give the enumeration of Turing machine
descriptions in such a way that we can effectively compute the
description of~$M$ from its index, and to effectively compute an index
of a machine~$M$ from its description.  By the Church--Turing thesis,
it is then possible to find a Turing machine which recovers the
description of the Turing machine with index~$e$ and writes the
corresponding description on its tape as output. The description would
be a sequence of blocks of~$\TMstroke$'s (representing the positive
integers in the sequence describing~$M_e$).

Given this, it now becomes natural to ask: what functions of Turing
machine indices are themselves computable by Turing machines? What
properties of Turing machine indices can be decided by Turing
machines?  An example: the function that maps an index~$e$ to the
number of states the Turing machine with index~$e$ has, is computable
by a Turing machine. Here's what such a Turing machine would do:
started on a tape containing a single block of $e$~$\TMstroke$'s, it
would first decode $e$ into its description. The description is now
represented by a sequence of blocks of~$\TMstroke$'s on the tape.
Since the first !!{element} in this sequence is the number of states.
So all that has to be done now is to erase everything but the first
block of $\TMstroke$'s and then halt.

A remarkable result is the following:

\begin{thm}\ollabel{thm:universal-tm} There is a \emph{universal
  Turing machine}~$U$ which, when started on input $\tuple{e,n}$ 
  \begin{enumerate}
    \item halts iff $M_e$ halts on input~$n$, and
    \item if $M_e$ halts with output $m$, so does~$U$.
  \end{enumerate}
  $U$ thus computes the function $f\colon \Nat \times \Nat \pto \Nat$
  given by $f(e,n) = m$ if $M_e$ started on input~$n$ halts with
  output~$m$, and undefined otherwise.
\end{thm}

\begin{proof}
  To actually produce~$U$ is basically impossible, since it is an
  extremely complicated machine. But we can describe in outline how it
  works, and then invoke the Church--Turing thesis.  When it starts,
  $U$'s tape contains a block of $e$ $\TMstroke$'s followed by a block
  of $n$~$\TMstroke$'s. It first ``decodes'' the index~$e$ to the
  right of the input~$n$. This produces a list of numbers (i.e.,
  blocks of $\TMstroke$'s separated by~$\TMblank$'s) that describes
  the instructions of machine~$M_e$. $U$ then writes the number of the
  start state of~$M_e$ and the number~$1$ on the tape to the right of
  the description of~$M_e$. (Again, these are represented in unary, as
  blocks of $\TMstroke$'s.) Next, it copies the input (block of
  $n$~$\TMstroke$'s) to the right---but it replaces each $\TMstroke$
  by a block of three $\TMstroke$'s (remember, the number of the $\TMstroke$
  symbol is~$3$, $1$ being the number of~$\TMendtape$ and $2$ being
  the number of~$\TMblank$). At the left end of this sequence of blocks
  (separated by $\TMblank$ symbols on the tape of~$U$), it writes a
  single~$\TMstroke$, the code for~$\TMendtape$.

  $U$ now has on its tape: the index~$e$, the number~$n$, the code
  number of the start state (the ``current state''), the number of the
  initial head position~$1$ (the ``current head position''), and the
  initial contents of the ``tape'' (a sequence of blocks
  of~$\TMstroke$'s representing the code numbers of the symbols
  of~$M_e$---the ``symbols''---separated by~$\TMblank$'s).

  It now simulates what $M_e$ would do if started on input~$n$, by
  doing the following:
  \begin{enumerate}
    \item Find the number~$k$ of the ``current head position'' (at the
    beginning, that's~$1$),
    \item Move to the $k$th block in the ``tape'' to see what the
    ``symbol'' there is,
    \item\ollabel{find-inst}%
    Find the instruction matching the current ``state'' and
    ``symbol,''
    \item Move back to the $k$th block on the ``tape'' and replace the
    ``symbol'' there with the code number of the symbol $M_e$ would
    write,
    \item Move the head to where it records the current ``state'' and
    replace the number there with the number of the new state,
    \item Move to the place where it records the ``tape position'' and
    erase a~$\TMstroke$ or add a~$\TMstroke$ (if the instruction says
    to move left or right, respectively).
    \item Repeat.\footnote{We're glossing over some subtle
    difficulties here. E.g., $U$~may need some extra space when it
    increases the counter where it keeps track of the ``current head
    position''---in that case it will have to move the entire ``tape''
    to the right.}
  \end{enumerate}
  If $M_e$ started on input~$n$ never halts, then $U$ also never
  halts, so its output is undefined.

  If in step~\olref{find-inst} it turns out that the description
  of~$M_e$ contains no instruction for the current
  ``state''/``symbol'' pair, then $M_e$ would halt. If this happens,
  $U$ erases the part of its tape to the left of the ``tape.'' For
  each block of three~$\TMstroke$'s (representing a~$\TMstroke$ on
  $M_e$'s tape), it writes a $\TMstroke$ on the left end of its own
  tape, and successively erases the ``tape.'' When this is done,
  $U$'s~tape contains a single block of~$\TMstroke$'s of length~$m$.
  
  If $U$ encounters something other than a block of
  three~$\TMstroke$'s on the ``tape,'' it immediately halts. Since
  $U$'s~tape in this case does not contain a single block
  of~$\TMstroke$'s, its output is not a natural number, i.e., $f(e,n)$
  is undefined in this case.
\end{proof}

\end{document}
