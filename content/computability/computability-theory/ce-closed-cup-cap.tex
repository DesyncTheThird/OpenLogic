% Part: computability
% Chapter: computability-theory
% Section: ce-closed-cup-cap

\documentclass[../../../include/open-logic-section]{subfiles}

\begin{document}

\olfileid{cmp}{thy}{clo}

\olsection[Union and Intersection of C.E. Sets]{Computably Enumerable
  Sets are Closed under Union and Intersection}

The following theorem gives some closure properties on the set of
computably enumerable sets.

\begin{thm}
Suppose $A$ and $B$ are computably enumerable. Then so are $A \cap B$
and $A \cup B$.
\end{thm}

\begin{proof}
\olref[eqc]{thm:ce-equiv} allows us to use various characterizations
of the computably enumerable sets. By way of illustration, we will
provide a few different proofs.

For the first proof, suppose $A$ is enumerated by a computable
function~$f$, and $B$ is enumerated by a computable function~$g$. Let
\begin{align*}
h(x) & = \umin{y}{(f(y) = x \lor g(y) = x)} \text{ and}\\
j(x) & = \umin{y}{(f((y)_0) = x \land g((y)_1) = x)}.
\end{align*}
Then $A \cup B$ is the domain of $h$, and $A \cap B$ is the domain
of~$j$.

\begin{explain}
Here is what is going on, in computational terms: given
procedures that enumerate $A$ and $B$, we can semi-decide if an
element $x$ is in $A \cup B$ by looking for $x$ in either enumeration;
and we can semi-decide if an element $x$ is in $A \cap B$ for looking
for $x$ in both enumerations at the same time.
\end{explain}

For the second proof, suppose again that $A$ is enumerated by~$f$ and
$B$ is enumerated by~$g$. Let
\[
k(x) = \begin{cases}
f(x/2) & \text{if $x$ is even} \\
g((x-1)/2) & \text{if $x$ is odd.}
\end{cases}
\]
Then $k$ enumerates $A \cup B$; the idea is that $k$ just alternates
between the enumerations offered by $f$ and~$g$. Enumerating $A \cap
B$ is tricker. If $A \cap B$ is empty, it is trivially computably
enumerable. Otherwise, let $c$ be any element of $A \cap B$, and
define $l$ by
\[
l(x) = \begin{cases}
f((x)_0) & \text{if $f((x)_0) = g((x)_1)$} \\
c & \text{otherwise.}
\end{cases}
\]
In computational terms, $l$ runs through pairs of elements in the
enumerations of $f$ and $g$, and outputs every match it finds;
otherwise, it just stalls by outputting $c$.

For the last proof, suppose $A$ is the \emph{domain} of the partial
function $m(x)$ and $B$ is the domain of the partial function
$n(x)$. Then $A \cap B$ is the domain of the partial function $m(x) +
n(x)$.

\begin{explain}
In computational terms, if $A$ is the set of values for which
$m$ halts and $B$ is the set of values for which $n$ halts, $A \cap B$
is the set of values for which both procedures halt.
\end{explain}

Expressing $A \cup B$ as a set of halting values is more difficult,
because one has to simulate $m$ and $n$ in parallel. Let $d$ be an
index for $m$ and let $e$ be an index for $n$; in other words, $m =
\cfind{d}$ and $n = \cfind{e}$. Then $A \cup B$ is the domain of the
function
\[
p(x) = \umin{y}{(T(d,x,y) \lor T(e,x,y))}.
\]
\begin{explain}
In computational terms, on input $x$, $p$ searches for either a
halting computation for $m$ or a halting computation for $n$, and
halts if it finds either one.
\end{explain}
\end{proof}

\end{document}
