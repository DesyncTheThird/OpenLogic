% Part: computability
% Chapter: computability-theory
% Section: halting-problem

\documentclass[../../../include/open-logic-section]{subfiles}

\begin{document}

\olfileid{cmp}{thy}{hlt}
\olsection{The Halting Problem}

By construction, the universal partial computable
function~$\fn{Un}(e,x)$ is defined if and only if the computation of
the function coded by~$e$ produces a value for input~$x$. It is
natural to ask if we can decide whether this is the case. In fact, it
is not. For the Turing machine model of computation, this means that
whether a given Turing machine halts on a given input is
computationally undecidable. The following theorem is therefore known
as the ``undecidability of the halting problem.'' We will provide two
proofs below. The first continues the thread of our previous
discussion, while the second is more direct.

\begin{thm}
\ollabel{thm:halting-problem}
Let
\[
h(e, x)  =
\begin{cases}
1 & \text{if\/ $\fn{Un}(e, x)$ is defined} \\
0 & \text{otherwise.}
\end{cases}
\]
Then $h$ is not computable.
\end{thm}

\begin{proof}
Suppose $h$ is computable. We show that this would let us define a
universal computable function. Define
\[
\fn{Un'}(e,x) =
\begin{cases}
\fn{Un}(e,x) & \text{if $h(e,x) = 1$} \\
0 & \text{otherwise.}
\end{cases}
\]
But now $\fn{Un'}(e, x)$ is a total function, and is computable if~$h$
is. For instance, we could define $g$ using primitive recursion, by
\begin{align*}
g(0, e, x) & \simeq 0 \\
g(y+1, e, x) & \simeq \fn{Un}(e,x);
\end{align*}
then
\[
\fn{Un'}(e,x) \simeq g(h(e,x),e,x).
\]
Since $\fn{Un'}(e,x)$ agrees with $\fn{Un}(e,x)$ wherever the latter
is defined, $\fn{Un'}$ is universal for those partial computable
functions that happen to be total. But this contradicts
\olref[nou]{thm:no-univ}.
\end{proof}

\begin{proof}
Suppose $h(e,x)$ were computable. Define the function $g$ by
\[
g(x) =
\begin{cases}
  0                & \text{if $h(x,x) = 0$} \\
  \fundefined & \text{otherwise.}
\end{cases}
\]
The function $g$ is partial computable. For example, one can define it
as $\umin{y}{h(x,x) = 0}$. So, for some~$e$, $g(x) \simeq \fn{Un}(e,
x)$ for every~$x$. Is $g$ defined at $e$?  If it is, then, by the
definition of~$g$, $h(e,e) = 0$ ($h$~can only take the value~$0$ if it
is defined). By the definition of~$h$, this means that $\fn{Un}(e, e)$
is undefined. By our assumption that $g(x) \simeq \fn{Un}(e, x)$
for every~$x$, we have that $g(e)$ is undefined, a contradiction.
On the other hand, if $g(e)$ is undefined, then $h(e,e) \neq 0$, and
so $h(e,e) = 1$. It follows that $\fn{Un}(e, e)$ is defined. But since
$g(x) \simeq \fn{Un}(e, x)$, then $g(e)$ would also be defined. Again,
a contradiction.
\end{proof}

\begin{tagblock}{TMs}
\begin{explain}
We can describe this argument in terms of Turing machines.  Suppose
there were a Turing machine~$H$ that takes as input a description of a
Turing machine~$E$ and an input~$x$, and decides whether or not $E$
halts on input~$x$. Then we could build another Turing machine~$G$
which takes a single input~$x$, runs $H$ to decide if the machine
$M_x$ with index~$x$ halts on input~$x$, and does the opposite. In
other words, if $H$ reports that $M_x$ halts on input~$x$, $G$ goes
into an infinite loop, and if $H$ reports that $M_x$ doesn't halt on
input~$x$, then $G$ just halts. Does $G$ halt on its own index as
input? The argument above shows that it does if and only if it
doesn't---a contradiction. So our supposition that there is a such
Turing machine~$H$ must be false.
\end{explain}
\end{tagblock}

\end{document}
