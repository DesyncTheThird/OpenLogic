% Part: intuitionistic-logic
% Chapter: tableaux
% Section: rules

\documentclass[../../../include/open-logic-section]{subfiles}

\begin{document}

\olfileid{int}{tab}{rul}

\olsection{Rules for Intuitionistic Logic}

The rules for the connectives $\land$ and $\lor$ are the same as for
regular propositional signed !!{tableau}s, just with prefixes added.
In each case, the rule applied to a signed !!{formula}
$\sFmla{S}{!A}[\sigma]$ produces new !!{formula}s that are also
prefixed by~$\sigma$. This should be intuitively clear: e.g., if $!A
\land !B$ is true at (a world named by)~$\sigma$, then $!A$ and $!B$
are true at~$\sigma$ (and not at any other world). We collect the
rules for $\land$ and $\lor$ in \olref{tab:prop-rules}.

\begin{table}
  \[\def\arraystretch{3}\begin{array}{|c|c|}
    \hline
    \AxiomC{\sFmla{\True}{!A \land !B}[\sigma]}
    \RightLabel{\TRule{\True}{\land}}
    \UnaryInfC{\sFmla{\True}{!A}[\sigma]}
    \noLine
    \UnaryInfC{\sFmla{\True}{!B}[\sigma]}
    \DisplayProof
    &
    \AxiomC{\sFmla{\False}{!A \land !B}[\sigma]}
    \RightLabel{\TRule{\False}{\land}}
    \UnaryInfC{$\sFmla{\False}{!A}[\sigma] \quad \mid \quad
      \sFmla{\False}{!B}[\sigma]$}
    \DisplayProof
    \\[2ex]
    \hline
    \AxiomC{\sFmla{\True}{!A \lor !B}[\sigma]}
    \RightLabel{\TRule{\True}{\lor}}
    \UnaryInfC{$\sFmla{\True}{!A}[\sigma] \quad \mid \quad
      \sFmla{\True}{!B}[\sigma]$}
    \DisplayProof
    &
    \AxiomC{\sFmla{\False}{!A \lor !B}[\sigma]}
    \RightLabel{\TRule{\False}{\lor}}
    \UnaryInfC{\sFmla{\False}{!A}[\sigma]}
    \noLine
    \UnaryInfC{\sFmla{\False}{!B}[\sigma]}
    \DisplayProof
    \\[2ex]
    \hline
  \end{array}\]
  \caption{Prefixed !!{tableau} rules for $\land$ and $\lor$}
  \ollabel{tab:prop-rules}
\end{table}

The closure condition is similar to that for ordinary !!{tableau}s,
although we require that not just the !!{formula}s, but also that the
prefixes must match. In fact, we can be somewhat more liberal: Since
in intuitionistic models, !!{formula}s, once true, remain true, it is
impossible that $!A$ is true at~$\sigma$ but false at any accessible
prefix~$\sigma.{*}$. So a branch is closed if it contains both
\[
\sFmla{\True}{!A}[\sigma] \quad\text{and}\quad \sFmla{\False}{!A}[\sigma.{*}]
\]
for some prefix $\sigma$ and !!{formula}~$!A$. Note that if the signs
are reversed, i.e., if it contains
\[
\sFmla{\False}{!A}[\sigma] \quad\text{and}\quad \sFmla{\True}{!A}[\sigma.{*}]
\]
the branch is closed only if $*$ is the empty sequence.

In addition, a branch is closed if it contains~$\sFmla{\True}{\bot}[\sigma]$.

The rules for setting up assumptions is also as for ordinary
!!{tableau}s, except that for assumptions we always use the
prefix~$1$. (It does not matter which prefix we use, as long as it's
the same for all assumptions.) So, e.g., we say that
\[
!B_1, \dots, !B_n \Proves !A
\]
iff there is a closed tableau for the assumptions
\[
\sFmla{\True}{!B_1}[1], \dots, \sFmla{\True}{!B_n}[1],
\sFmla{\False}{!A}[1].
\]

For the conditional~$\lif$, the rules differ from the classical and
modal cases. The $\TRule{\lif}{\True}$ rule extends a branch
containing $\sFmla{\True}{!A \lif !B}[\sigma]$ by
$\sFmla{\True}{!A}[\sigma.{*}]$ and $\sFmla{\False}{!B}[\sigma.{*}]$ on two
different branches. It can only be applied for a prefix~$\sigma.{*}$
which \emph{already} occurs on the branch in which it is applied.
Let's call such a prefix ``used'' (on the branch). (Since $\sigma.{*}$
includes $\sigma$ itself, the rule can always be applied by adding the
prefixed signed formulas $\sFmla{\True}{!A}[\sigma]$ and
$\sFmla{\False}{!B}[\sigma]$ on separate branches.)

The $\TRule{\lif}{\False}$ rule extends a branch containing
$\sFmla{\False}{!A \lif !B}[\sigma]$ by both
$\sFmla{\True}{!A}[\sigma.n]$ and $\sFmla{\False}{!B}[\sigma.n]$ on
the same branch, with $\sigma.n$ a prefix new to the branch. 

The rules for $\lnot$ are defined analogously (using the definition of
$\lnot !A$ as $!A \lif \lfalse$).

The rules are given in \olref{tab:rules-lif-lnot}.

\begin{table}
  \[\def\arraystretch{3}\begin{array}{|c|c|}
    \hline
    \AxiomC{\sFmla{\True}{\lnot !A}[\sigma]}
    \RightLabel{\TRule{\True}{\lnot}}
    \UnaryInfC{$\sFmla{\False}{!A}[\sigma.{*}]$}
    \DisplayProof
    &
    \AxiomC{\sFmla{\False}{\lnot !A}[\sigma]}
    \RightLabel{\TRule{\False}{\lnot}}
    \UnaryInfC{\sFmla{\True}{!A}[\sigma.n]}
    \DisplayProof
    \\[1ex]
    \text{$\sigma.{*}$ is used} & \text{$\sigma.n$ is new}\\
    \hline
    \AxiomC{\sFmla{\True}{!A \lif !B}[\sigma]}
    \RightLabel{\TRule{\True}{\lif}}
    \UnaryInfC{$\sFmla{\False}{!A}[\sigma.{*}] \quad \mid \quad
      \sFmla{\True}{!B}[\sigma.{*}]$}
    \DisplayProof
    &
    \AxiomC{\sFmla{\False}{!A \lif !B}[\sigma]}
    \RightLabel{\TRule{\False}{\lif}}
    \UnaryInfC{\sFmla{\True}{!A}[\sigma.n]}
    \noLine
    \UnaryInfC{\sFmla{\False}{!B}[\sigma.n]}
    \DisplayProof
    \\[1ex]
    \text{$\sigma.{*}$ is used} & \text{$\sigma.n$ is new}\\
    \hline
  \end{array}\]
  \caption{Prefixed !!{tableau} rules for $\lnot$ and $\lif$}
  \ollabel{tab:rules-lif-lnot}
\end{table}

\end{document}
