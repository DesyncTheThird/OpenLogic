% Part: lambda-calculus
% Chapter: introduction
% Section: beta

\documentclass[../../../include/open-logic-section]{subfiles}

\begin{document}

\olfileid{lam}{int}{bet}
\olsection{$\beta$-reduction}

When we see $(\lambd[m][(\lambd[y][y]) m])$, it is natural to
conjecture that it has some connection with $\lambd[m][m]$, namely the
second term should be the result of ``simplifying'' the first.  The
notion of \emph{$\beta$-reduction} captures this intuition formally.

\begin{defn}[$\beta$-contraction, $\bredone$] \ollabel{defn:betacontr}
  \emph{The $\beta$-contraction} ($\bredone$) is the smallest compatible
  relation on terms satisfying the following condition:
  \[
    (\lambd[x][N])Q \bredone \Subst{N}{Q}{x} 
  \]
  We say $P$ is \emph{$\beta$-contracted} to $Q$ if $P \bredone Q$.  A
  term of the form $(\lambd[x][N])Q$ is called a \emph{redex}.
\end{defn}

\begin{prob} \ollabel{prob:def}
  Spell out the equivalent inductive definitions of $\beta$-contraction as we
  did for change of bound variable in \olref[lam][syn][alp]{defn:aconvone}.
\end{prob}
  
\begin{defn}[$\beta$-reduction, $\bred$] \ollabel{defn:betared}
  \emph{$\beta$-reduction} ($\bred$) is the smallest reflexive,
  transitive relation on terms containing $\bredone$.  We say $P$ is
  $\beta$-reduced to $Q$ if $P \bred Q$.
\end{defn}

We will write $\redone$ instead of $\bredone$, and $\red$ instead of
$\bred$ when context is clear.

Informally speaking, $M \bred N$  if and only if $M$ can be changed to
$N$ by zero or several steps of $\beta$-contraction.

\begin{defn}[$\beta$-normal]
A term that cannot be $\beta$-contracted any further is said to be
\emph{$\beta$-normal}. 
\end{defn}

If $M \bred N$ and $N$ is $\beta$-normal, then we say $N$ is a
\emph{normal form} of~$M$. One may ask if the normal form of a term is
unique, and the answer is yes, as we will see later.

Let us consider some examples.
\begin{enumerate}
\item We have
\begin{align*}
(\lambd[x][xxy]) \lambd[z][z] & \redone (\lambd[z][z])(\lambd[z][z]) y \\
& \redone (\lambd[z][z]) y \\
& \redone y
\end{align*}
\item ``Simplifying'' a term can actually make it more complex:
\begin{align*}
(\lambd[x][xxy])(\lambd[x][xxy]) & \redone (\lambd[x][xxy])(\lambd[x][xxy])y \\
& \redone (\lambd[x][xxy])(\lambd[x][xxy])yy \\
& \redone \dots
\end{align*}
\item It can also leave a term unchanged:
\[
(\lambd[x][xx])(\lambd[x][xx]) \redone (\lambd[x][xx])(\lambd[x][xx])
\]
\item Also, some terms can be reduced in more than one way; for
  example,
\[
(\lambd[x][(\lambd[y][yx]) z]) v \redone (\lambd[y][yv]) z
\]
by contracting the outermost application; and
\[
(\lambd[x][(\lambd[y][yx]) z]) v \redone (\lambd[x][zx]) v
\]
by contracting the innermost one. Note, in this case, however, that
both terms further reduce to the same term, $zv$.
\end{enumerate}

The final outcome in the last example is not a coincidence, but rather
illustrates a deep and important property of the lambda calculus, known as the
Church--Rosser property.

\begin{digress}
  In general, there is more than one way to $\beta$-reduce a term,
  thus many reduction strategies have been invented, among which the
  most common is the \emph{natural strategy}.  The natural strategy
  always contracts the \emph{left-most} redex, where the position of a
  redex is defined as its starting point in the term.  The natural
  strategy has the useful property that a term can be reduced to a
  normal form by some strategy iff it can be reduced to normal form
  using the natural strategy. In what follows we will use the natural
  stratuegy unless otherwise specified.
\end{digress}

\begin{defn}[$\beta$-equivalence, $\equal$]
  \emph{$\beta$-Equivalence} ($\equal$) is the relation inductively
  defined as follows:
  \begin{enumerate}
  \item $M \equal M$.
  \item If $M \equal N$, then $N \equal M$.
  \item If $M \equal N$, $N \equal O$, then $M \equal O$.
  \item If $M \equal N$, then $PM \equal PN$.
  \item If $M \equal N$, then $MQ \equal NQ$.
  \item If $M \equal N$, then $\lambd[x][M] \equal \lambd[x][N]$.
  \item $(\lambd[x][N])Q \equal \Subst{N}{Q}{x}$.
  \end{enumerate}
\end{defn}

The first three rules make the relation an equivalence relation; the
next three make it compatible; the last ensures that it contains
$\beta$-contraction.

Informally speaking, two terms are $\beta$-equivalent if and only if
one of them can be changed to the other in zero or more steps of
$\beta$-contraction, or ``inverse'' of $\beta$-contraction. The inverse of
$\beta$-contraction is defined so that $M$ inverse-$\beta$-contracts to $N$
iff $N$ $\beta$-contracts to $M$.

Besides the above rules, we will extend the relation with more
rules, and denote the extended equivalence relation as $\equal[X]$,
where $X$ is the extending rule.

\end{document}
