\documentclass[../../../include/open-logic-section]{subfiles}

\begin{document}

\olfileid{sth}{z}{power}
\olsection{Powersets}

We will proceed with another axiom:

\begin{axiom}[Powersets]
For any set $A$, the set $\Pow{A} = \Setabs{x}{x \subseteq A}$ exists.
\[
	\forall A \exists P \forall x(x \in P \liff (\forall z \in x)z \in A)
\]
\end{axiom}

Our justification for this is pretty straightforward. Suppose $A$ is
formed at stage $S$. Then all of $A$'s members were available before
$S$ (by \stagesacc). So, reasoning as in our justification for
Separation, every subset of $A$ is formed by stage $S$. So they are
all available, to be formed into a single set, at any stage after $S$.
And we know that there is some such stage, since $S$ is not the last
stage (by \stagessucc). So $\Pow{A}$ exists.

Here is a nice consequence of Powersets:

\begin{prop}\label{thm:Products}
Given any sets $A, B$, their Cartesian product $A \times B$ exists.
\end{prop}

\begin{proof}
The set $\Pow{\Pow{A \cup B}}$ exists by Powersets and
\olref[sth][z][pairs]{prop:pairsconsequences}. So by Separation, this
set exists:
\[
	C = \Setabs{z \in \Pow{\Pow{A \cup B}}}{(\exists x \in A)(\exists y \in B) z = \tuple{x, y}}.
\]
Now, for any $x \in A$ and $y \in B$, the set $\tuple{x, y}$ exists by
\olref[sth][z][pairs]{prop:pairsconsequences}. Moreover, since $x, y
\in A \cup B$, we have that $\{x\}, \{x, y\} \in \Pow{A \cup B}$, and
$\tuple{x,y} \in \Pow{\Pow{A \cup B}}$. So $A \times B = C$.
\end{proof}

In this proof, Powerset interacts with Separation. And that is no
surprise. Without Separation, Powersets wouldn't be a very
\emph{powerful} principle. After all, Separation tells us which
subsets of a set exist, and hence determines just how ``fat'' each
Powerset is.

\begin{prob}
Show that, for any sets $A, B$: (i) the set of all relations with
domain $A$ and range $B$ exists; and (ii) the set of all functions
from $A$ to $B$ exists.
\end{prob}

\begin{prob}
Let $A$ be a set, and let $\sim$ be an equivalence relation on $A$.
Prove that the set of equivalence classes under $\sim$ on $A$, i.e.,
$\equivclass{A}{\sim}$, exists.
\end{prob}

\end{document}
