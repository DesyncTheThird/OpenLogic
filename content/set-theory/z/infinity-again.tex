\documentclass[../../../include/open-logic-section]{subfiles}

\begin{document}

\olfileid{sth}{z}{infinity-again}	
\olsection{Infinity}

We already have enough axioms to ensure that there are infinitely many
sets (if there are any). For suppose some set exists, and so
$\emptyset$ exists (by \olref[sth][z][sep]{prop:emptyexists}). Now for
any set~$x$, the set $x \cup \{x\}$ exists by
\olref[sth][z][pairs]{prop:pairsconsequences}. So, applying this a few
times, we will get sets as follows:
\begin{enumerate}
	\item[0.] $\emptyset$ %& $0$ & stage $0$\\
	\item[1.] $ \{\emptyset\}$ %& $1$ & stage $1$\\
	\item[2.] $\{\emptyset, \{\emptyset\}\}$ %& $2$ & stage $2$\\
	\item[3.] $\{\emptyset, \{\emptyset\}, \{\emptyset, \{\emptyset\}\}\}$ %&$3$ & stage $3$\\
	\item[4.] $\{\emptyset, \{\emptyset\}, \{\emptyset, \{\emptyset\}\}, \{\emptyset, \{\emptyset\}, \{\emptyset, \{\emptyset\}\}\}\}$% & $4$ & stage $4$ 
\end{enumerate}
and we can check that each of these sets is distinct. 

We have started the numbering from $0$, for a few reasons. But one of
them is this. It is not that hard to check that the set we have
labelled ``$n$'' has exactly $n$ members, and (intuitively) is formed
at the $n$th stage. 

But. This gives us \emph{infinitely many} sets, but it does not
guarantee that there is an \emph{infinite set}, i.e., a set with
infinitely many members. And this really matters: unless we can find a
(Dedekind) infinite set, we cannot construct a Dedekind algebra. But
we want a Dedekind algebra, so that we can treat it as the set of
natural numbers. (Compare
\olref[sfr][infinite][dedekindsproof]{sec}.)

Importantly, the axioms we have laid down so far do \emph{not}
guarantee the existence of any infinite set. So we have to lay down a
new axiom:

\begin{axiom}[Infinity]
There is a set, $I$, such that $\emptyset \in I$ and $x \cup \{x\} \in I$ whenever $x \in I$.
\begin{align*}
	\exists I( & (\exists o \in I)\forall x\ x \notin o \land {}\\
	& (\forall x \in I)(\exists s \in I)\forall z(z \in s \liff (z \in x \lor z = x)))
\end{align*}
\end{axiom}

It is easy to see that the set $I$ given to us by the Axiom of
Infinity is Dedekind infinite. Its distinguished element is
$\emptyset$, and the injection on $I$ is given by $s(x) = x\cup
\{x\}$. Now,
\olref[sfr][infinite][dedekind]{thm:DedekindInfiniteAlgebra} showed
how to extract a Dedekind Algebra from a Dedekind infinite set; and we
will treat this as our set of natural numbers. More precisely:
\begin{defn}\ollabel{defnomega}
Let $I$ be any set given to us by the Axiom of Infinity. Let $s$
be the function $s(x) = x \cup \{x\}$. Let $\omega =
\closureofunder{s}{\emptyset}$. We call the members of $\omega$ the
\emph{natural numbers}, and say that $n$ is the result of $n$-many
applications of $s$ to $\emptyset$.
\end{defn}

You can now look back and check that the set labelled ``$n$'', a few
paragraphs earlier, will be treated \emph{as} the number $n$. 

We will discuss this significance of this stipulation in
\olref[sth][z][nat]{sec}. For now, it enables us to prove an
intuitive result:

\begin{prop}\ollabel{naturalnumbersarentinfinite}
No natural number is Dedekind infinite.
\end{prop}

\begin{proof}
The proof is by induction, i.e.,
\olref[sfr][infinite][induction]{thm:dedinfiniteinduction}. Clearly $0
= \emptyset$ is not Dedekind infinite. For the induction step, we will
establish the contrapositive: if (absurdly) $s(n)$ is Dedekind
infinite, then $n$ is Dedekind infinite. 

So suppose that $s(n)$ is Dedekind infinite, i.e., there is some
!!{injection} $f$ with $\ran{f}\subsetneq \dom{f} = s(n) = n \cup
\{n\}$. There are two cases to consider. 

\emph{Case 1: $n \notin \ran{f}$.} So $\ran{f} \subseteq n$, and $f(n) \in n$. Let $g = \funrestrictionto{f}{n}$; now $\ran{g} =
\ran{f} \setminus \{f(n)\} \subsetneq n = \dom{g}$. Hence $n$ is
Dedekind infinite. 

\emph{Case 2: $n \in \ran{f}$.} Fix $m \in \dom{f} \setminus \ran{f}$, and define a function $h$ with domain $s(n) = n \cup \{n\}$:
\[
h(x) = 
	\begin{cases}
		f(x) & \text{if }f(x) \neq n\\
		m & \text{if }f(x)=n
	\end{cases}
\]
So $h$ and $f$ agree everywhere, except that $h(f^{-1}(n)) = m
\neq n = f(f^{-1}(n))$. Since $f$ is !!a{injection}, $n \notin
\ran{h}$; and $\ran{h} \subsetneq \dom{h} = s(n)$. Now $n$ is
Dedekind infinite, using the argument of Case 1.
\end{proof}

The question remains, though, of how we might \emph{justify} the Axiom
of Infinity. The short answer is that we will need to add another
principle to the story we have been telling. That principle is as
follows:
\begin{enumerate}
	\item[] \stagesinf. There is an infinite stage. That is, there is a stage which (a) is not the first stage, and which (b) has some stages before it, but which (c) has no immediate predecessor.
\end{enumerate}
The Axiom of Infinity follows straightforwardly from this principle.
We know that natural number $n$ is formed at stage $n$. So the set
$\omega$ is formed at the first infinite stage. And $\omega$ itself
witnesses the Axiom of Infinity. 

This, however, simply pushes us back to the question of how we might
justify \stagesinf. As with \stagessucc, it was not an explicit part
of the story we told about the cumulative-iterative hierarchy. But
more than that: nothing in the very idea of an iterative hierarchy, in
which sets are formed stage by stage, forces us to think that the
process involves an \emph{infinite} stage. It seems perfectly coherent
to think that the stages are ordered like the natural numbers. 

This, however, gives rise to an obvious problem. In
\olref[sfr][infinite][dedekindsproof]{sec}, we considered
Dedekind's ``proof'' that there is a Dedekind infinite set (of
thoughts). This may not have struck you as very satisfying. But if
\stagesinf{} is not ``forced upon us'' by the iterative conception of
set (or by ``the laws of thought''), then we are still left without an
intrinsic justification for the claim that there is a Dedekind
infinite set.

There is much more to say here, of course. But hopefully you are now
at a point to start thinking about what it might \emph{take} to
justify an axiom (or principle). In what follows we will simply take
\stagesinf{} for granted.

\end{document}