\documentclass[../../../include/open-logic-section]{subfiles}

\begin{document}

\olfileid{sth}{story}{blv}

\olsection{Appendix: Frege's Basic Law V}

In \olref[rus]{sec}, we explained that Russell's formulated his
paradox as a problem for the system Frege outlined in his
\emph{Grundgesetze}. Frege's system did not include a direct
formulation of Na\"{i}ve Comprehension. So, in this appendix, we will
very briefly explain what Frege's system \emph{did} include, and how
it relates to Na\"ive Comprehension and how it relates to Russell's
Paradox.

Frege's system is \emph{second-order}, and was designed to formulate
the notion of an \emph{extension of a concept}.\footnote{Strictly
speaking, Frege attempts to formalize a more general notion: the
``value-range'' of a function. Extensions of concepts are a special
case of the more general notion. See \citet[pp.\ 8--9]{Heck2012} for
the details.} Using notation inspired by Frege, we will write
$\fregeext{x}{F(x)}$ for \emph{the extension of the concept~$F$}. This
is a device which takes a \emph{predicate}, ``$F$'', and turns it into
a (first-order) \emph{term}, ``$\fregeext{x}{F(x)}$''. Using this
device, Frege offered the following \emph{definition} of membership:
\[
	a \in b =_\text{df} \exists G(b = \fregeext{x}{G(x)} \land Ga)
\]
roughly: $a \in b$ iff $a$ falls under a concept whose extension is
$b$. (Note that the quantifier ``$\exists G$'' is second-order.) Frege
also maintained the following principle, known as \emph{Basic Law V}: 
$$\fregeext{x}{F(x)} = \fregeext{x}{G(x)} \liff \forall x (Fx \liff Gx)$$
roughly: concepts have identical extensions iff they are coextensive. (Again, both ``$F$'' and ``$G$'' are in predicate position.) Now a simple principle connects membership with property-satisfaction:

\begin{lem}[in \emph{Grundgesetze}]\ollabel{lem:Fregeextensions}
$\forall F \forall a(a \in \fregeext{x}{F(x)} \liff Fa)$
\end{lem}

\begin{proof} 
Fix $F$ and $a$. Now $a \in \fregeext{x}{F(x)}$ iff $\exists G(\fregeext{x}{F(x)}
= \fregeext{x}{G(x)} \land Ga)$ (by the definition of membership) iff
$\exists G(\forall x(Fx \liff Gx) \land Ga)$ (by Basic Law V) iff $Fa$
(by elementary second-order logic).
\end{proof}

And this yields Na\"ive Comprehension almost immediately:

\begin{lem}[in \emph{Grundgesetze}.]
$\forall F \exists s \forall a (a \in s \liff Fa)$
\end{lem}

\begin{proof}
Fix $F$; now \olref{lem:Fregeextensions} yields $\forall a (a \in
\fregeext{x}{F(x)} \liff Fa)$; so $\exists s\forall a(a \in s \liff Fa)$ by
existential generalisation. The result follows since $F$ was
arbitrary.
\end{proof}

Russell's Paradox follows by taking $F$ as given by $\forall x(Fx \liff x \notin x)$. 

\end{document}